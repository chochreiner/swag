\documentclass[a4paper]{article}
%\usepackage[cm]{fullpage}
\usepackage[utf8]{inputenc}
\usepackage[T1]{fontenc}
\usepackage{lmodern}
\usepackage[english]{babel}
\usepackage{graphicx}
%\usepackage{scrpage2}
%\usepackage{ifthen}
%\usepackage[ruled,lined]{algorithm2e}
%\usepackage{amsmath}
%\usepackage{amsthm}
%\usepackage{amssymb}
%\usepackage{units}
\usepackage{enumerate}
\usepackage{multirow}
\usepackage{fancyhdr}
\usepackage{rotating}
\usepackage{pdfpages}
\usepackage{lscape}

\begin{document}
\pagestyle{fancy}
\fancyfoot{}
\fancyhead{}
\fancyfoot[L]{SWA2011}
\fancyfoot[C]{\thepage}
\fancyfoot[R] {Group 30}
\renewcommand{\headrulewidth}{0.0pt}

\begin{center}
\includegraphics[width=1.0\textwidth]{logo}

\vspace{3cm}

\huge \textsf{\textbf{Assignment 3\\Implementation and Documentation}}

\vspace{1ex}

\Large \rm{Software Architectures}

\vspace{1ex}

\Large \rm{Summer Term 2011}

\vspace{1ex}

\normalsize Christoph Hochreiner, 0726292

\normalsize Fabian Gruber, 0726905

\normalsize Felix Rinker, 0726724

\normalsize Tobias Hochwallner, 0557189

\end{center}
\pagebreak


\tableofcontents


\pagebreak

\section{Deployment Guide}

\subsection{Prerequesits}

\paragraph{Database}
The system uses a PostgreSQL 8.4 database with the name \emph{swa} and a corresponding user with the username \emph{swa} and the password \emph{swa11}.

\paragraph{Application Server}
The prototype uses the Glassfish 3.1 application server for deploying the different components ant the Glassfish JMS implementation for the communication between these components. Therefore the $GLASSFISH\_HOME$ environment variable has to be set.

\paragraph{Build and setup the environment}
\begin{verbatim}
ant build
\end{verbatim}

The ant build target executes different tasks in order to prepare and build the prototype. These tasks cover the generation of models, creation of JMS resources and compiled the code. 

%TODO: insert single tasks that are excuted for the run task


\paragraph{Execute the prototype}
\begin{verbatim}
ant run
\end{verbatim}

The ant run target deploys the single components and the application can be accessed via \emph{http://localhost:8080/swag}.

%TODO: insert single tasks that are excuted for the build task


\section{Final Architecture}


\subsection{System Architecture}

\subsubsection{swag-model}

\subsubsection{swag-messages}

\subsubsection{swag-util}

\subsubsection{swag-auth}

\subsubsection{swag-executor}

\subsubsection{swag-webapp}

\subsubsection{swag-notification}

\subsubsection{mailserver}


\subsection{Technology stack}
\paragraph{Apache Wicket}\footnote{http://wicket.apache.org/}
The Apache Wicket Framework was used for the creation of the Web application.

\paragraph{Guice}\footnote{http://code.google.com/p/google-guice/}
Guice was used for the dependency injection.

\paragraph{Quartz Scheduler} \footnote{http://www.quartz-scheduler.org/}
For the implementation of our delayed actions, we integrated the quartz scheduler, that provides the necessary functionality to schedule tasks, that are executed to specific dates in the future.

\paragraph{JMS}
The JMS technology was used for the communication between the components. For this prototype, we relied on the JMS implementation of the Glassfish Server.

\paragraph{Maven}
For development purposes we used maven in order to build, setup and deploy the applications. In order to meet the requirements, we transformed our maven targets to ant with <TODO insert Plugin>



%TODO add addtionial technologies



 






\section{Possible Improvements}

\end{document}
